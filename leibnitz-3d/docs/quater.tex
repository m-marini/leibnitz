\documentclass[a4paper,11pt]{article}
\usepackage[T1]{fontenc}
\usepackage[utf8]{inputenc}
\usepackage{lmodern}
\usepackage[italian]{babel}
\usepackage{graphicx}
\newtheorem{theorem}{Teorema}

\title{Studio matematico per software Leibnitz}
\author{Marco Marini}

\begin{document}

\maketitle
\tableofcontents

\begin{abstract}
\end{abstract}

\part{Quaternioni}

\section{Introduzione ai quaternioni}
Sia a $H$ l'insieme di elementi della forma
\begin{equation}
	\label{eq:quatForm}
	a = a_1+a_2 i +a_3 j +a_4 k = a_1 + \vec a
\end{equation}
Definiamo l'operazione $+$ come
\begin{equation}
\begin{array}{r}
	a = a_1 + a_2 i + a_3 j + a_4 k,  b = b_1 + b_2 i + b_3 j + b_4 k
	\\
	a + b = (a_1 + b_1) + (a_2 + b_2) i + (a_3 + b_3) j + (a_4 + b_4) k
\end{array}
\end{equation}
o in notazione mista
\begin{equation}
\begin{array}{r}
	\label{eq:somma}
	a = a_1 + \vec a,  b = b_1 + \vec b
	\\
	a + b = (a_1 + b_1) + (\vec a + \vec b)
\end{array}
\end{equation}
e il prodotto $\cdot$ come il normale prodotto con le regole di moltiplicazione tra letterali:
\begin{equation}
\begin{array}{r}
	i \cdot i = j \cdot j = k \cdot k = -1
	\\
	i \cdot j = k, j \cdot k = i, 	k \cdot i = j 
	\\
	j \cdot i = -k, k \cdot j = -i, i \cdot k = -j 
\end{array}
\end{equation}
per cui
\begin{equation}
\begin{array}{r}
	a \cdot b  =a_1 b_1 - a_2 b_2 - a_3 b_3 - a_4 b_4 +
\\
	+ (a_1 b_2 + a_2 b_1 + a_3 b_4 - a_4 b_3) i +
\\
	+ (a_1 b_3 - a_2 b_4 + a_3 b_1 + a_4 b_2) j +
\\
	+ (a_1 b_4 + a_2 b_3 - a_3 b_2 + a_4 b_1) k
\end{array}
\end{equation}
oppure
\begin{equation}
\begin{array}{r}
	\label{eq:prodotto}
	a \cdot b = (a_1 b_1 - \vec a \vec b) + (a_1 \vec b + \vec a b_1 + \vec a \times \vec b)
\end{array}
\end{equation}

\begin{theorem}
L'insieme $H$ è un corpo rispetto a (\ref{eq:somma}) e (\ref{eq:prodotto})
\end{theorem}

\paragraph{Dimostrazione.}
Dimostriamo che $H$ è un gruppo rispetto alla somma:

Dalla (\ref{eq:somma}) si vede che $a+b$ è della forma (\ref{eq:quatForm}) quindi $\forall a, b \in H: a + b \in H$ quindi $H$ è chiuso rispetto $+$.

Abbiamo poi
\begin{eqnarray*}
	(a + b) + c  = [(a_1 + \vec a) + (b_1 + \vec b)] + (c_1 + \vec c)
\\
	= [a_1 + b_1 + (\vec a + \vec b)]+ (c_1 + \vec c)
\\
	= a_1 + b_1 + c_1 + [(\vec a + \vec b) + \vec c]
\end{eqnarray*}
per la proprietà associativa dei reali in $\Re$ e dei vettori in $\Re^3$ abbiamo
\begin{eqnarray*}
	a_1 + b_1 + c_1 + [(\vec a + \vec b) + \vec c]  = 
\\
	a_1 + (b_1 + c_1) + [\vec a + (\vec b + \vec c)]
\\
	=( a_1 + \vec a) + [(b_1 + \vec b) + (c_1 + \vec c)]
\\
	=a + (b + c)
\end{eqnarray*}
quindi la somma in $H$ è associativa

Prendiamo l'elemento $0 = (0 + \vec 0)$
abbiamo che
\[
a + 0 = 0 + a = a, \forall a \in H
\]

Prendiamo l'elemento $-a = (-a - \vec a)$
abbiamo che 
\[
a + (-a) = (-a) + a = 0, \forall a \in H
\]

Quindi $H$ è un gruppo come volevasi dimostrare.

Dimostriamo che $H$ è un gruppo rispetto al prodotto:

Dalla (\ref{eq:prodotto}) si vede che $a \cdot b$ è della forma (\ref{eq:quatForm}) quindi $\forall a, b \in H: a, b \in H$ quindi $H$ è chiuso rispetto $\cdot$.

Abbiamo poi
\begin{eqnarray*}
	(a b) c  = [(a_1 + a_2 i + a_3 j + a_4 k) (b_1 + b_2 i + b_3 j + b_4 k)] (c_1 + c_2 i + c_3 j + c_4 k)
\\
\\
	= [(a_1 b_1 - a_2 b_2 - a_3 b_3  - a_4 b_4)  + (a_1 b_2 + a_2 b_1 + a_3 b_4 - a_4 b_3) i
\\
	+ (a_1 b_3 + a_3 b_1 + a_4 b_2 - a_2 b_4) j + (a_1 b_4 + a_4 b_1 + a_2 b_3 - a_3 b_2) k]
\\
	(c_1 + c_2 i + c_3 j + c_4 k)
\\
\\
	= [(a_1 b_1 - a_2 b_2 - a_3 b_3  - a_4 b_4) c_1  - (a_1 b_2 + a_2 b_1 + a_3 b_4 - a_4 b_3) c_2
\\
	- (a_1 b_3 + a_3 b_1 + a_4 b_2 - a_2 b_4) c_3 - (a_1 b_4 + a_4 b_1 + a_2 b_3 - a_3 b_2) c_4]
\\
	+ [(a_1 b_1 - a_2 b_2 - a_3 b_3  - a_4 b_4) c_2 + (a_1 b_2 + a_2 b_1 + a_3 b_4 - a_4 b_3) c_1
\\
	+(a_1 b_3 + a_3 b_1 + a_4 b_2 - a_2 b_4) c_4 -  (a_1 b_4 + a_4 b_1 + a_2 b_3 - a_3 b_2) c_3] i
\\
	+ [(a_1 b_1 - a_2 b_2 - a_3 b_3  - a_4 b_4) c_3 - (a_1 b_2 + a_2 b_1 + a_3 b_4 - a_4 b_3) c_4
\\
	+(a_1 b_3 + a_3 b_1 + a_4 b_2 - a_2 b_4) c_1 +  (a_1 b_4 + a_4 b_1 + a_2 b_3 - a_3 b_2) c_2] j
\\
	+ [(a_1 b_1 - a_2 b_2 - a_3 b_3  - a_4 b_4) c_4 + (a_1 b_2 + a_2 b_1 + a_3 b_4 - a_4 b_3) c_3
\\
	-(a_1 b_3 + a_3 b_1 + a_4 b_2 - a_2 b_4) c_2 +  (a_1 b_4 + a_4 b_1 + a_2 b_3 - a_3 b_2) c_1] k
\\
\\
	= [a_1 (b_1 c_1 -b_2 c_2- b_3 c_3 - b_4 c_4) - a_2 (b_1 c_2 + b_2 c_1 + b_3 c_4 - b_4 c_3)
\\
	- a_3 (b_1 c_3 + b_3 c_1 + b_4 c_2 - b_2 c_4) - a_4 (b_1 c_4 + b_4 c_1 + b_2 c_3 - b_3 c_2)]
\\
	+ [a_2 (b_1 c_1 -b_2 c_2- b_3 c_3 - b_4 c_4) + a_1 (b_1 c_2 + b_2 c_1 + b_3 c_4 - b_4 c_3)
\\
	- a_4 (b_1 c_3 + b_3 c_1 + b_4 c_2 - b_2 c_4) + a_3 (b_1 c_4 + b_4 c_1 + b_2 c_3 - b_3 c_2)] i
\\
	+ [a_3 (b_1 c_1 -b_2 c_2- b_3 c_3 - b_4 c_4) + a_4 (b_1 c_2 + b_2 c_1 + b_3 c_4 - b_4 c_3)
\\
	+ a_1 (b_1 c_3 + b_3 c_1 + b_4 c_2 - b_2 c_4) - a_2  (b_1 c_4 + b_4 c_1 + b_2 c_3 - b_3 c_2)] j
\\
	+ [a_4 (b_1 c_1 -b_2 c_2- b_3 c_3 - b_4 c_4) - a_3 (b_1 c_2 + b_2 c_1 + b_3 c_4 - b_4 c_3)
\\
	+ a_2 (b_1 c_3 + b_3 c_1 + b_4 c_2 - b_2 c_4) + a_1  (b_1 c_4 + b_4 c_1 + b_2 c_3 - b_3 c_2)] k
\\
\\
	= (a_1 + a_2 i + a_3 j + a_4 k) [(b_1 + b_2 i + b_3 j + b_4 k)] (c_1 + c_2 i + c_3 j + c_4 k)]  = a (b c)
\end{eqnarray*}
quindi il prodotto è associativo.

Prendiamo l'elemento $a^{-1} = \frac{\bar a}{|a|^2} = \frac{a1 - a_2 i - a_3 j - a_4 k}{a_1^2 + a_2^2 + a_3^2 + a_4^2 } $ con $a \ne 0$ abbiamo che
\[
	a \cdot a^{-1} = a^{-1} \cdot a = 1, \forall a \ne 0, a \in H
\]

Quindi  esiste l'elemento inverso e $H$ e un gruppo rispetto al prodotto e quindi è un corpo.


Sviluppo di $(a b) c$ per coefficienti:
\begin{eqnarray*}
	(a b) c = [(a_1 b_1 - a_2 b_2 - a_3 b_3  - a_4 b_4) c_1  - (a_1 b_2 + a_2 b_1 + a_3 b_4 - a_4 b_3) c_2
\\
	- (a_1 b_3 + a_3 b_1 + a_4 b_2 - a_2 b_4) c_3 - (a_1 b_4 + a_4 b_1 + a_2 b_3 - a_3 b_2) c_4]
\\
	+ [(a_1 b_1 - a_2 b_2 - a_3 b_3  - a_4 b_4) c_2 + (a_1 b_2 + a_2 b_1 + a_3 b_4 - a_4 b_3) c_1
\\
	+(a_1 b_3 + a_3 b_1 + a_4 b_2 - a_2 b_4) c_4 -  (a_1 b_4 + a_4 b_1 + a_2 b_3 - a_3 b_2) c_3] i
\\
	+ [(a_1 b_1 - a_2 b_2 - a_3 b_3  - a_4 b_4) c_3 - (a_1 b_2 + a_2 b_1 + a_3 b_4 - a_4 b_3) c_4
\\
	+(a_1 b_3 + a_3 b_1 + a_4 b_2 - a_2 b_4) c_1 +  (a_1 b_4 + a_4 b_1 + a_2 b_3 - a_3 b_2) c_2] j
\\
	+ [(a_1 b_1 - a_2 b_2 - a_3 b_3  - a_4 b_4) c_4 + (a_1 b_2 + a_2 b_1 + a_3 b_4 - a_4 b_3) c_3
\\
	-(a_1 b_3 + a_3 b_1 + a_4 b_2 - a_2 b_4) c_2 +  (a_1 b_4 + a_4 b_1 + a_2 b_3 - a_3 b_2) c_1] k
\\
\\
	= [a_1 b_1 c_1
\\
	- (a_2 b_2 + a_3 b_3 + a_4 b_4) c_1 - (a_2 c_2 + a_3 c_3 + a_4 c_4) b_1 - (b_2 c_2 + b_3 c_3 + b_4 c_4) a_1
\\
	- (a_3 b_4 - a_4 b_3) c_2 - (a_4 b_2 - a_2 b_4) c_3 - (a_2 b_3 - a_3 b_2) c_4]
\\
	+ a_1 b_1(c_2 i + c_3 j + c_4 k) + a_1 (b_2 i + b_3 j + b_4 k) c_1 + (a_2 i + a_3 j + a_4 k) b_1 c_1
\\
	+ a_1 [(b_3 c_4 - b_4 c_3) i + (b_4 c_2 - b_2 c_4) j + (b_2 c_3 - b_3 c_2) k]
\\
	+ b_1 [(a_3 c_4 - a_4 c_3) i + (a_4 c_2 - a_2 c_4) j + (a_2 c_3 - a_3 c_2) k]
\\
	+ c_1 [(a_3 b_4 - a_4 b_3) i + (a_4 b_2 - a_2 b_4) j + (a_2 b_3 - a_3 b_2) k]
\\
	- (a_2 b_2 + a_3 b_3 + a_4 b_4)(c_2 i + c_3 j + c_4 k)
\\
	+(a_4 b_2 c_4 - a_2 b_4 c_4 - a_2 b_3 c_3 + a_3 b_2 c_3) i
\\
	+( -a_3 b_4 c_4 + a_4 b_3 c_4 + a_2 b_3 c_2 - a_3 b_2 c_2) j
\\
	+ (a_3 b_4 c_3 - a_4 b_3 c_3 -a_4 b_2 c_2- a_2 b_4 c_2) k
\\
\\
	= a_1 b_1 c_1 - \left(\vec a \vec b \right) c_1 - \left(\vec a \vec c \right) b_1 - \left(\vec b \vec c \right) a_1 - \left (\vec a \times \vec b \right) \vec c
\\
	+ a_1 b_1 \vec c + a_1 c_1 \vec b + b_1 c_1 \vec a + a_1 (\vec b \times \vec c) + b_1 (\vec a \times \vec c) + c_1 (\vec a \times \vec b)
\\
	- \left( \vec a \vec b \right) \vec c + \left( \vec a \times \vec b \right) \times \vec c
\end{eqnarray*}

Sviluppo di $(a b) c$ per vettori:
\begin{eqnarray*}
	(a b) c = [(a_1 + \vec a)(b_1 + \vec b)] (c_1 + \vec c)
\\
	= \left[a_1 b_1 - \left( \vec a \vec b \right) + a_1 \vec b + b_1 \vec a + \left( \vec a \times \vec b \right) \right]
	(c_1 + \vec c)
\\
	= a_1 b_1 c_1 - \left( \vec a \vec b \right) c_1 - a_1 \left( \vec b \vec c \right)  - b_1 \left( \vec a \vec c \right) - \left( \vec a \times \vec b \right) \vec c
\\
	+ a_1 b_1 \vec c - \left( \vec a \vec b \right) \vec c + a_1 c_1 \vec b +b_1 c_1 \vec a
	 + c_1 \left( \vec a \times \vec b \right)
\\
	 + a_1 \left( \vec b \times \vec c \right)
	 + b_1 \left( \vec a \times \vec c \right) + \left( \vec a \times \vec b \right) \times \vec c
\\
	a (b c)
	= a_1 b_1 c_1 - \left( \vec a \vec b \right) c_1 - a_1 \left( \vec b \vec c \right)  - b_1 \left( \vec a \vec c \right) - \vec a \left( \vec b \times \vec c \right)
\\
	+ a_1 b_1 \vec c - \vec a \left( \vec b \vec c \right) + a_1 c_1 \vec b +b_1 c_1 \vec a
	 + c_1 \left( \vec a \times \vec b \right)
\\
	 + a_1 \left( \vec b \times \vec c \right)
	 + b_1 \left( \vec a \times \vec c \right) + \vec a \times \left( \vec b \times \vec c \right)
\end{eqnarray*}

\section{Quaternioni e rotazioni.}
Sia 
\begin{equation}
	q = e^{\frac{\varphi}{2} (u_x i + u_y j + u_z k)} = e^{\frac{\varphi}{2} \vec u} = \cos \frac{\varphi}{2} + \sin \frac{\varphi}{2} (u_x i + u_y j + u_z k)= \cos \frac{\varphi}{2} + \sin \frac{\varphi}{2} \vec u
\end{equation}
con $|\vec u|^2 = u_x^2+u_y^2+u_z^2= 1$ allora
\begin{equation}
	q^{-1} = \bar q = e^{-\frac{\varphi}{2} \vec u} = \cos \frac{\varphi}{2} - \sin \frac{\varphi}{2} \vec u
\end{equation}
quindi
\begin{equation}
	|q|^2 = q \bar q = q q^{-1}= e^0 = 1
\end{equation}
quindi $q$ è un quaternione unitario.

\begin{theorem}
Se $q = e^{\frac{\varphi}{2} \vec a} = \cos \frac{\varphi}{2} + \sin \frac{\varphi}{2} \vec a$ e $v = - \bar v = \vec v$,
la trasformazione
\begin{equation}
	\alpha_q : p \rightarrow v' = q v q^{-1}
\end{equation}
rappresenta una rotazione dello spazio euclideo $\Re^3$ di $\varphi$ attorno all'asse $\vec u = (u_x, u_y, u_z)$
\end{theorem}

\paragraph{Dimostrazione.}
Sviluppiamo la trasformazione:
\begin{eqnarray*}
	v' = q v q^{-1} = \left(\cos \frac{\varphi}{2} + \sin \frac{\varphi}{2} \vec u\right) \vec v \left(\cos \frac{\varphi}{2} - \sin \frac{\varphi}{2} \vec u\right)
\\
	= -\sin \frac{\varphi}{2} \cos \frac{\varphi}{2} (\vec u \vec v)
	+\sin \frac{\varphi}{2} \cos \frac{\varphi}{2} (\vec u \vec v)
	- \sin^2 \frac{\varphi}{2} (\vec u \times \vec v) \vec u
\\
	+ \sin^2 \frac{\varphi}{2} (\vec u \vec v) \vec u + \cos^2 \frac{\varphi}{2} \vec v
	+ \sin \frac{\varphi}{2} \cos \frac{\varphi}{2} (\vec u \times \vec v) 
	- \sin \frac{\varphi}{2} \cos \frac{\varphi}{2} (\vec v \times \vec u) 
\\
	- \sin^2  \frac{\varphi}{2} (\vec u \times \vec v) \times \vec u
\end{eqnarray*}
essendo
\begin{eqnarray*}
	(\vec u \times \vec v) \vec u = 0
\\
	\vec v \times \vec u = -\vec u \times \vec v
\end{eqnarray*}
abbiamo
\begin{eqnarray*}
	v' = 
	\sin^2 \frac{\varphi}{2} (\vec u \vec v) \vec u + \cos^2 \frac{\varphi}{2} \vec v
	+ 2 \sin \frac{\varphi}{2} \cos \frac{\varphi}{2} (\vec u \times \vec v) 
	- \sin^2  \frac{\varphi}{2} (\vec u \times \vec v) \times \vec u
\end{eqnarray*}
ma scomponendo il vettore $\vec v$ nelle componenti $\vec v_\parallel$ parallela a $\vec u$  e $\vec v_\perp$ perpendicolare a $\vec u$
\begin{eqnarray*}
	 \vec v = \vec v_\parallel +  \vec v_\perp 
\\
	(\vec u \vec v) \vec u = \vec v_\parallel
\\
	\vec u \times \vec v_\parallel = 0
\\
	\vec u \times \vec v = \vec u \times \vec v_\parallel + \vec u \times \vec v_\perp = \vec u \times \vec v_\perp
\\
	(\vec u \times \vec v) \times \vec u = (\vec u \times \vec v_\perp) \times \vec u = \vec v_\perp
\end{eqnarray*}
risulta
\begin{eqnarray*}
	v' = 
	\sin^2 \frac{\varphi}{2} \vec v_\parallel + \cos^2 \frac{\varphi}{2} (\vec v_\parallel + \vec v_\perp)
	+ 2 \sin \frac{\varphi}{2} \cos \frac{\varphi}{2} (\vec u \times \vec v_\perp) 
	- \sin^2  \frac{\varphi}{2} \vec v_\perp
\\
	= 
	\vec v_\parallel + \left(\cos^2 \frac{\varphi}{2} - \sin^2 \frac{\varphi}{2} \right) \vec v_\perp
	+ 2 \sin \frac{\varphi}{2} \cos \frac{\varphi}{2} (\vec u \times \vec v_\perp) 
\\
	= 
	\vec v_\parallel + \left(1 - 2 \sin^2 \frac{\varphi}{2} \right) \vec v_\perp
	+ 2 \sin \frac{\varphi}{2} \cos \frac{\varphi}{2} (\vec u \times \vec v_\perp) 
\end{eqnarray*}
applicando le formule di bisezione
\begin{eqnarray*}
	2 \sin \frac{\varphi}{2} \cos \frac{\varphi}{2} = \sin \varphi
\\
	2 \sin^2 \frac{\varphi}{2} = 1 - \cos \varphi
\end{eqnarray*}
abbiamo
\begin{eqnarray*}
	v' = \vec v_\parallel + (1 - 1 + cos \varphi) \vec v_\perp + \sin \varphi (\vec u \times \vec v_\perp) 
\end{eqnarray*}
da cui
\begin{equation}
	\label{eq:rotate}
	v' = \vec v_\parallel + \cos \varphi \vec v_\perp + \sin \varphi (\vec u \times \vec v_\perp) 
\end{equation}
che è la formula di una rotazione di $\varphi$ attorno all'asse $\vec u$.

\end{document}