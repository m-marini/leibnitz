\documentclass[a4paper,11pt]{article}
\usepackage[T1]{fontenc}
\usepackage[utf8]{inputenc}
\usepackage{lmodern}
\usepackage[italian]{babel}
\usepackage{graphicx}
\newtheorem{theorem}{Teorema}

\title{Studio matematico per software Leibnitz}
\author{Marco Marini}

\begin{document}

\maketitle
\tableofcontents

\begin{abstract}
\end{abstract}

\part{Calcolo tensoriale}

\section{Vettore controvariante}

Prendiamo un sistema cartesiano associato ad un sistema di riferimento $ x^1, \dots, x^n $.
Prendiamo poi un diverso sistema sistema di riferimento con basi $ z^1, \dots, z^n $ e le relative funzioni
di trasformazione dei due sistema di riferimento
\begin{eqnarray*}
  z^1=z^1(x^1, \dots, x^n)
  \\
  \dots
  \\
  z^n=z^n(x^1, \dots, x^n)
\end{eqnarray*}
.
 
Prendiamo una funzione parametrica che esprime una linea nel sistema $ x_1, \dots, x_n $:
\[ f(t) = ( f^1(t), \dots,  f^n(t) ) \] 
La stessa funzione è espressa nel sistema $ z^1, \dots, z^n $ come:
\[ h(t) = ( h^1(t), \dots,  h^n(t) ) \]
ovvero
\begin{eqnarray*}
  h^1(t) = z^1 \left( f^1(t), \dots, f^n(t) \right)
  \\
  \dots
  \\
  h^n(t) = z^n \left( f^1(t), \dots, f^n(t) \right)
\end{eqnarray*}

Calcoliamo la derivata rispetto $ t $
\begin{equation}\label{eq:vettore1}
  \frac{d}{dt} f(t) = \left( \frac{d}{dt} f^1(t), \dots, \frac{d}{dt} f^n(t) \right)
\end{equation}
mentre nel sistema di riferimento $ (x'_1, \dots, x'_n) $ sarà:
\begin{equation}\label{eq:vettore2}
  \frac{d}{dt} h(t)= \left( \frac{d}{dt} h^1(t), \dots, \frac{d}{dt} h^n(t) \right)
\end{equation}
applicando le funzioni di trasformazione avremo
\begin{equation}\label{eq:vettore3}
\begin{array}{r}
  \frac{d}{dt} h^1(t) = \sum_{i=1}^n \frac{\partial z^1}{\partial x^i} \frac{d}{dt} f^i(t)
  \\
  \dots
  \\
  \frac{d}{dt} h^n= \sum_{i=1}^n \frac{\partial z^n}{\partial x^i} \frac{d}{dt} f^i(t)
\end{array}
\end{equation}

Siano $ \vec {e_1}, \dots , \vec {e_n} $ e $ \vec {e_1'}, \dots , \vec {e_n'} $ i vettori base
dei rispettivi sistemi di riferimento. Possiamo esprimere la (\ref{eq:vettore1})
con la notazione vettoriale:
\[ \frac{d}{dt} \vec f(t) = \frac{d}{dt} f^1(t) \vec {e_1} + \dots + \frac{d}{dt} f^n(t) \vec {e_n} = \sum_{i=1}^n \frac{d}{dt} f^i(t) \vec {e_i}\]
o in forma più concisa (notazione di Einstein)
\[ \frac{d}{dt} \vec f(t) = \frac{d}{dt} f^i(t) \vec {e_i} \]

Allo stesso modo possiamo esprimere la (\ref{eq:vettore2}) come:
\[ \frac{d}{dt} \vec h(t) = \frac{d}{dt} h^i(t) \vec {e_i'} \]
e dalla (\ref{eq:vettore3})
\[ \frac{d}{dt} \vec h(t) = \frac{\partial z^i}{\partial x^j} \frac{d}{dt} f^j(t) \vec {e_i'} \]

La notazione vettoriale permette di identificare facilmente il sistema di riferimento evidenziando le specifiche basi.

\paragraph{Esempio}
Prendiamo il sistema di riferimento euclideo $ x, y, z $ e il sistema cilindrico $ r, \psi , z $.
Le funzioni di trasformazione sono:
\begin{eqnarray*}
  x = r \cos \psi
  \\
  y = r \sin \psi
  \\
  z = z
\end{eqnarray*}

Prendiamo la curva $ f(t) $ espressa nel sistema $r, \psi , z $ dalle espressioni
\[
  f(t) = ( t, \psi_0,  k t)
\]

La curva è una retta passante per l'origine con pendenza $k$ e diretta con angolo polare $ \psi_0 $.

Nelle coordinare euclidee la curva è:
\[
  h(t) = (t \cos \psi_0, t \sin \psi_0, k t )
\]

La derivata nel sistema $ r, \psi , z $ è
\[ \frac{d}{dt} \vec f(t) = \frac{d}{dt} f^i(t) \vec {e_i}
  = \vec {e_r} + 0 \vec {e_\psi} + k \vec {e_z}
\].

Posto che
\[
\begin{array}{rrr}
  \frac{\partial x}{\partial r} = \cos \psi\ & \frac{\partial x}{\partial \psi} = -r \sin \psi & \frac{\partial x}{\partial z} = 0
  \\
  \frac{\partial y}{\partial r} = \sin \psi\ & \frac{\partial x}{\partial \psi} = r \cos \psi & \frac{\partial x}{\partial z} = 0
  \\
  \frac{\partial z}{\partial r} = 0\ & \frac{\partial z}{\partial \psi} = 0 & \frac{\partial z}{\partial z} = 1
\end{array}
\]

Nel sistema $ x, y, z $ avremo:
\begin{eqnarray*}
  \frac{d}{dt} \vec h(t)= \frac{\partial z^i}{\partial x^j} \frac{d}{dt} f^j(t) \vec {è_i}
  \\
  \\
  = \left( \frac{\partial x}{\partial r} \frac{d}{dt} f^r(t) + \frac{\partial x}{\partial \psi} \frac{d}{dt} f^\psi(t) + \frac{\partial x}{\partial z} \frac{d}{dt} f^z(t) \right) \vec {e_x}
  \\
  + \left( \frac{\partial y}{\partial r} \frac{d}{dt} f^r(t) + \frac{\partial y}{\partial \psi} f^\psi(t) + \frac{\partial y}{\partial z} \frac{d}{dt} f^z(t) \right) \vec {e_y}
  \\
  + \left( \frac{\partial z}{\partial r} \frac{d}{dt} f^r(t) + \frac{\partial z}{\partial \psi} f^\psi(t) + \frac{\partial z}{\partial z} \frac{d}{dt} f^z(t) \right) \vec {e_z}
  \\
  \\
  = \cos \psi \vec {e_x} + \sin \psi \vec {e_y} + k \vec {e_z}
\end{eqnarray*}
ma essendo $ \psi = \psi_0 $ abbiamo 
\[
  \frac{d}{dt} \vec h(t)= \cos \psi_0 \vec {e_x} + \sin \psi_0 \vec {e_y} + k \vec {e_z}
\]

\subsection{Generalizzazione}

Sia $\xi = (\xi ^1, \dots , \xi ^n)$ un vettore associato al sistema di coordinate $ x^1, \dots , x^n $.

Se due sistemi di coordinate $ x^1, \dots , x^n $ e $ z^1, \dots , z^n $ sono legati mediante una trasformazione della forma $ x= x(z) $ tale che $ x^i (z_0^1, \dots , z_0^n) = x_0^i, i = 1, \dots , n $, lo stesso vettore si definisce nel nuovo sistema di coordinate $ z $ mediante un altro insieme di punti 
$ \zeta ^1, \dots, \zeta ^n $ legato dalla formula:
\begin{equation}\label{eq:trasformaVettore}
  \xi ^i = \frac{\partial x^i}{\partial z^j} \zeta ^j
\end{equation}

Se $ \vec {e_i} = (\vec {e_1}, \dots, \vec {e_n}) $ e $ \vec {e_i'} = (\vec {e_1'}, \dots, \vec {e_n'}) $ sono i vettori base rispettivamente in $ x^1, \dots , x^n $ e $  z^1, \dots , z^n $ abbiamo:
\begin{eqnarray*}
  \vec \zeta = \zeta^i \vec {e_i'}
  \\
  \vec \xi = \xi^i \vec {e_i} = \frac{\partial x^i}{\partial z^j} \zeta ^j \vec {e_i}
\end{eqnarray*}
.

\subsection{Trasformazione inversa}
Sia $\xi = (\xi ^1, \dots , \xi ^n)$ un vettore associato al sistema di coordinate $ x^1, \dots , x^n $ e 
$\zeta = (\zeta ^1, \dots , \zeta ^n)$ lo stesso vettore associato al sistema di coordinate $ z^1, \dots , z^n $.
per la (\ref{eq:trasformaVettore}) abbiamo:
\[
  \xi ^i = \frac{\partial x^i}{\partial z^j} \zeta ^j
\]

Identifichiamo le relative funzioni di trasformazione inverse da $ z^1, \dots , z^n $ a  $ x^1, \dots , x^n $  con
\begin{eqnarray*}
  x^1=x^1(z^1, \dots, z^n)
  \\
  \dots
  \\
  x^n=x^n(z^1, \dots, z^n)
\end{eqnarray*}
avremo che 
\[
  \zeta ^i = \frac{\partial z^i}{\partial x^j} \xi ^j
\]
quindi
\[
  \zeta ^i = \frac{\partial z^i}{\partial x^j} \frac{\partial x^j}{\partial z^k} \zeta ^k = \delta^i_k \zeta ^k
\].

Da questo consegue che
\begin{equation}
\frac{\partial z^i}{\partial x^j} \frac{\partial x^j}{\partial z^k} =  \delta^i_k
\Longrightarrow
\frac{\partial z^i}{\partial x^j} = \left( \frac{\partial x^i}{\partial z^j} \right)^{-1}
\end{equation}


\section{Vettore covariante}

Prendiamo un sistema cartesiano associato ad un sistema di riferimento $ x^1, \dots, x^n $.
Prendiamo poi un diverso sistema sistema di riferimento con coordinate $ z^1, \dots, z^n $ e le relative funzioni
di trasformazione dei due sistema di riferimento
\begin{eqnarray*}
  z^1=z^1(x^1, \dots, x^n)
  \\
  \dots
  \\
  z^n=z^n(x^1, \dots, x^n)
\end{eqnarray*}
.
 
Prendiamouna funzione scalare nel sistema $ x_1, \dots, x_n $ 
\[ f(x^1, \dots, x^n) \]

La stessa funzione è espressa nel sistema $ z_1, \dots, z_n $ come:
\[  h(z^1, \dots,  z^n) = f(x^1, \dots, x^n) \]
ovvero
\[
  h(z^1 ( x^1, \dots, x^n), \dots, z^n ( x^1, \dots, x^n) )
\]

Calcoliamo il gradiente della funzione:
\begin{equation}\label{eq:covettore1}
 	grad (f) = \left( \frac{\partial f}{\partial x^1} , \dots, \frac{\partial f}{\partial x^n} \right)
 	= \left( \frac{\partial h}{\partial x^1} , \dots, \frac{\partial f}{\partial x^n} \right)
\end{equation} 
mentre nel sistema di riferimento $ (x'_1, \dots, x'_n) $ sarà:
\begin{equation}\label{eq:covettore2}
	grad (h)= \left( \frac{\partial h}{\partial z^1}, \dots, \frac{\partial h}{\partial z^n} \right)
\end{equation}
applicando le funzioni di trasformazione avremo
\begin{equation}\label{eq:covettore3}
\begin{array}{r}
	grad (f) = \left(
	 \frac{\partial h(z^1 ( x^1, \dots, x^n), \dots, z^n ( x^1, \dots, x^n)}{\partial x^1} ,
	\dots,
	\frac{\partial h(z^1 ( x^1, \dots, x^n), \dots, z^n ( x^1, \dots, x^n)}{\partial x^n} 
	\right)
\\
	= \left(
	 \sum \frac{\partial h}{\partial z^i} \frac{\partial z^i}{\partial x^1}
	\dots,
	 \sum \frac{\partial h}{\partial z^i} \frac{\partial z^i}{\partial x^n}
	\right)
\end{array}
\end{equation}

Siano $ (\vec {e^1}, \dots , \vec {e^n}) $ e $ (\vec {{e'}^1}, \dots , \vec {{e'}^n}) $
i vettori base dei rispettivi
sistemi di riferimento. Possiamo esprimere la (\ref{eq:covettore1}) con la notazione vettoriale:
\[ grad (f) = \frac{\partial f}{\partial x^1 } \vec {e^1} + \dots + \frac{\partial f}{\partial x^n} \vec {e^n}
	= \sum_{i=1}^n \frac{\partial f}{\partial x^i} \vec {e^i}
\]
o in forma più concisa (notazione di Einstein)
\[ grad(f) = \frac{\partial f}{\partial x^i} \vec {e^i} \]

Allo stesso modo possiamo esprimere la (\ref{eq:covettore2}) come:
\[ grad (h) = \frac{\partial h}{\partial z^i} \vec {{e'}^i} \]
e dalla (\ref{eq:covettore3})
\[ grad (f) = \frac{\partial h}{\partial z^j} \frac{\partial z^j}{\partial x^i}  \vec {{e'}_i} \]

La notazione vettoriale permette di identificare facilmente il sistema di riferimento evidenziando le specifiche basi.

\paragraph{Esempio}
Prendiamo il sistema di riferimento euclideo $ x, y, z $ e il sistema cilindrico $ r, \psi , z $.
Le funzioni di trasformazione sono:
\begin{eqnarray*}
  x = r \cos \psi
  \\
  y = r \sin \psi
  \\
  z = z
\end{eqnarray*}

Prendiamo la funzione $ h(x, y, z) $ espressa nel sistema $ x, y, z $ dalle espressioni
\[
  h(x, y, z) =  k_1 x +k_2 y + k_3 z
\]

Nel sistema cilindrico la funzione diventa:
\[
  f( r, \psi, z) =  k_1 r \cos \psi + k_2 r \sin \psi + k_3 z
\]

Il gradiente nel sistema $ x, y, z $ è
\[ grad(h) = \frac{\partial h}{\partial z^i} \vec {e^i}
  = k_1 \vec {e^x} + k_2 \vec {e^y} + k_3 \vec {e^z}
\].

Posto che
\[
\begin{array}{rrr}
  \frac{\partial x}{\partial r} = \cos \psi\ & \frac{\partial x}{\partial \psi} = -r \sin \psi & \frac{\partial x}{\partial z} = 0
  \\
  \frac{\partial y}{\partial r} = \sin \psi\ & \frac{\partial x}{\partial \psi} = r \cos \psi & \frac{\partial x}{\partial z} = 0
  \\
  \frac{\partial z}{\partial r} = 0\ & \frac{\partial z}{\partial \psi} = 0 & \frac{\partial z}{\partial z} = 1
\end{array}
\]

Nel sistema $ r, \psi, z $ avremo:
\begin{eqnarray*}
	grad (f) = \frac{\partial z^j}{\partial x^i} \frac{\partial h}{\partial z^j} \vec {{e'}^i}
  \\
  \\
	= \left( \frac{\partial x}{\partial r} \frac{\partial h}{\partial x} + \frac{\partial y}{\partial r} \frac{\partial h}{\partial y} + \frac{\partial z}{\partial r} \frac{\partial h}{\partial z} \right) \vec {e^r}
  \\
	+ \left( \frac{\partial x}{\partial \psi} \frac{\partial h}{\partial x} + \frac{\partial y}{\partial \psi} \frac{\partial h}{\partial y} + \frac{\partial z}{\partial \psi} \frac{\partial h}{\partial z} \right) \vec {e^\psi}
  \\
	+ \left( \frac{\partial x}{\partial z} \frac{\partial h}{\partial x} + \frac{\partial y}{\partial z} \frac{\partial h}{\partial y} + \frac{\partial z}{\partial z} \frac{\partial h}{\partial z} \right) \vec {e^z}
\\
\\
	= (k_1 \cos \psi + k_2 \sin \psi) \vec {e^r} + (-k_1 r \sin \psi + k_2 r \cos \psi) \vec {e^\psi} + k_z \vec {e^z}
\end{eqnarray*}


\subsection{Generalizzazione}

Sia $\xi = (\xi_1, \dots , \xi_n)$ un covettore associato al sistema di coordinate $ x^1, \dots , x^n $.

Se due sistemi di coordinate $ x^1, \dots , x^n $ e $  z^1, \dots , z^n  $ sono legati mediante una trasformazione della forma $ x= x(z) $ tale che $ x^i (z_0^1, \dots , z_0^n) = x_0^i, i = 1, \dots , n $, lo stesso covettore si definisce nel nuovo sistema di coordinate $ z $ mediante un altro insieme di punti 
$ \zeta_1, \dots, \zeta_n $ legato dalla formula:
\begin{equation}\label{eq:trasformaCovettore}
  \zeta_i = \frac{\partial x^j}{\partial z^i} \xi_j
\end{equation}

Se $ \vec {e^i} = (\vec {e^1}, \dots, \vec {e^n}) $ e $ \vec {{e'}^i} = (\vec { {e'}^1}, \dots, \vec {{e'}_n}) $ sono i vettori base rispettivamente in $ x^1, \dots , x^n $ e $ z^1, \dots , z^n $ abbiamo:
\begin{eqnarray*}
  \vec \xi = \xi_i \vec {e^i}
  \\
  \vec \zeta = \zeta_i \vec {{e'}^i} = \frac{\partial x^j}{\partial z^i} \zeta_j \vec {{e'}^i}
\end{eqnarray*}
.

\subsection{Trasformazione inversa}

Sia $\xi = (\xi_1, \dots , \xi_n)$ un covettore associato al sistema di coordinate $ x^1, \dots , x^n $ e 
$\zeta = (\zeta_1, \dots , \zeta_n)$ lo stesso covettore associato al sistema di coordinate $ z^1, \dots , z^n $.
per la (\ref{eq:trasformaCovettore}) abbiamo:
\[
  \zeta_i = \frac{\partial x^j}{\partial z^i} \xi_j
\]

Identifichiamo le relative funzioni di trasformazione inverse da $ z^1, \dots , z^n $ a  $ x^1, \dots , x^n $  con
\begin{eqnarray*}
  x^1=x^1(z^1, \dots, z^n)
  \\
  \dots
  \\
  x^n=x^n(z^1, \dots, z^n)
\end{eqnarray*}
avremo che 
\[
  \xi_i = \frac{\partial z^j}{\partial x^i} \zeta_j
\]
quindi
\[
  \xi_i = \frac{\partial z^j}{\partial x^i} \frac{\partial x^k}{\partial z^j}  \xi_k = \delta _i^k \xi _k
\].

Da questo consegue che
\begin{equation}
	\frac{\partial z^j}{\partial x^i} \frac{\partial x^k}{\partial z^j} = \delta _i^k
\Longrightarrow
	\frac{\partial z^j}{\partial x^i} = \left( \frac{\partial x^i}{\partial z^j} \right)^{-1}
\end{equation}


\section{Metrica}

Prendiamo due curve rappresentate dalle funzione parametriche $ f_1(t), f_2(t) $ espresse nel
sistema $ x^1, \dots, x^n $.

Consideriamo gli elementi infinitesimali lineari sulle curve considerate nello stesso punto $ t = t_0 $
(distanze infinitesimale tra due punti ):
\begin{eqnarray*}
	\vec {dx_1} = dx_1^i \vec e_i = \left. \frac{d f^i_1}{dt} dt \, \vec e_i \right|_{t=t_0}
\\
	\vec {dx_2} = dx_2^i \vec e_i = \left. \frac{d f^i_2}{dt} dt \, \vec e_i \right|_{t=t_0}
\end{eqnarray*}

Supponiamo che il sistema $ x^1, \dots, x^n $ sia euclideo.
Abbiamo che il prodotto scalare dei due vettori è
\begin{equation}
	\vec{dx_1} \vec{dx_2} = dx_1^i dx_2^i = \delta_{ij} dx_1^i dx_2^j = \delta_{ij} \frac{d f^i_1}{dt} \frac{d f^j_2}{dt} dt^2
\end{equation}

Siano $ h_1(t), h_2(t) $ la relative funzioni espresse nel sistema $ z^1, \dots, z^n $.
Consideriamo gli elementi infinitesimali lineari sulle curve considerate nello stesso punto $ t = t_0 $:
\begin{eqnarray*}
	\vec {dz_1} = dz_1^i \vec e'_i = \left. \frac{d h^i_1}{dt} dt \, \vec e'_i \right|_{t=t_0}
\\
	\vec {dz_2} = dz_2^i \vec e'_i = \left. \frac{d h^i_2}{dt} dt \, \vec e'_i \right|_{t=t_0}
\end{eqnarray*}

Applicando le trasformazioni abbiamo che
\begin{eqnarray*}
	\vec {dx_1} = \frac{\partial x^i}{\partial z^j}  \frac{d h^j_1}{dt} dt \, \vec e_i =  \frac{\partial x^i}{\partial z^j}  dz_1^j dt \, \vec e_i 
\\
	\vec {dx_2} = \frac{\partial x^i}{\partial z^j}  \frac{d h^j_2}{dt} dt \, \vec e_i = \frac{\partial x^i}{\partial z^j} dz^j_2 dt \, \vec e_i 
\end{eqnarray*}
quindi il prodotto scalare può essere espresso con
\[
	\vec {ds_1} \vec {ds_2} = \delta_{ij} \frac{\partial x^i}{\partial z^k} dz^k_1 \frac{\partial x^j}{\partial z^l} dz^l_2
\]
ponendo
\[
	g_{ij} = \delta_{kl} \frac{\partial x^k}{\partial z_i} \frac{\partial x^l}{\partial z^j}
\]
abbiamo
\[
	\vec {ds_1} \vec {ds_2} = g_{ij} dz^k_1 dz^l_2
\]

Il prodotto scalare è invariante rispetto al sistema di riferimento quindi
\[
\vec {ds_1} \vec {ds_2} = \vec {ds'_1} \vec {ds'_2}
\]
ma viene espresso da forme bilineari diverse nei vari sistemi di riferimento:
\[
\vec {ds_1} \vec {ds_2} = \vec {ds'_1} \vec {ds'_2}  = \delta_{ij} dx_1^i dx_2^j = g_{ij} dz_1^i dz_2^j
\]

Le forme bilineare $\delta_{ij} dx_1^i dx_2^j $ e $ g_{ij} dz_1^i dz_2^j $ si dicono metriche.
Se $ g_{ij} $ è la metrica nel sistema $ x^1, \dots, x^n $ e $g'_{ij} $ è la metrica nel sistema $ z^1, \dots, z^n $, le matrice metriche si trasformano secondo le funzioni:
\begin{equation}
	g'_{ij} = g_{kl} \frac{\partial x^k}{\partial z^i} \frac{\partial x^l}{\partial z^j}
\end{equation}

Si può notare che le matrici metriche sono matrici simmetriche quindi
\[
	g_{ij} = g_{ji}
\]

\paragraph{Esempio}

Nel sistema euclideo la metrica è
\[
	g_{ij} = \delta_{ij} =
	\left|
\begin{array}{ccc}
	1 & 0 & 0
\\
	0 & 1 & 0
\\
	0 & 0 & 1
\end{array}
	\right|
\]

Nel sistema cilindrico $ z= (r, \psi, z) $ invece è:
\[
	g'_{ij} = \delta_{kl} \frac{\partial x^k}{\partial z^i} \frac{\partial x^l}{\partial z^j} = \frac{\partial x^k}{\partial z^i} \frac{\partial x^k}{\partial z^j}
\]
cioè
\begin{eqnarray*}
	g'_{ij} = \left|
\begin{array}{ccc}
	(\frac{\partial x}{\partial r})^2 + (\frac{\partial y}{\partial r})^2 + (\frac{\partial z}{\partial r})^2,
	& \frac{\partial x}{\partial r}\frac{\partial x}{\partial \psi} + \frac{\partial y}{\partial r}\frac{\partial y}{\partial \psi} + \frac{\partial z}{\partial r}\frac{\partial z}{\partial \psi},
	& \frac{\partial x}{\partial r}\frac{\partial x}{\partial z} + \frac{\partial y}{\partial r}\frac{\partial y}{\partial z} + \frac{\partial z}{\partial r}\frac{\partial z}{\partial z}
\\
	\frac{\partial x}{\partial r}\frac{\partial x}{\partial \psi} + \frac{\partial y}{\partial r}\frac{\partial y}{\partial \psi} + \frac{\partial z}{\partial r}\frac{\partial z}{\partial \psi},
	& (\frac{\partial x}{\partial \psi})^2 + (\frac{\partial y}{\partial \psi})^2 + (\frac{\partial z}{\partial \psi})^2,
	& \frac{\partial x}{\partial \psi}\frac{\partial x}{\partial z} + \frac{\partial y}{\partial \psi}\frac{\partial y}{\partial z} + \frac{\partial z}{\partial \psi}\frac{\partial z}{\partial z}
\\
	\frac{\partial x}{\partial r}\frac{\partial x}{\partial z} + \frac{\partial y}{\partial r}\frac{\partial y}{\partial z} + \frac{\partial z}{\partial r}\frac{\partial z}{\partial z},
	& \frac{\partial x}{\partial \psi}\frac{\partial x}{\partial z} + \frac{\partial y}{\partial \psi}\frac{\partial y}{\partial z} + \frac{\partial z}{\partial \psi}\frac{\partial z}{\partial z},
	& (\frac{\partial x}{\partial z})^2 + (\frac{\partial y}{\partial z})^2 + (\frac{\partial \psi}{\partial z})^2
\end{array}
	\right|
\\
	= \left|
\begin{array}{ccc}
	\cos^2 \psi + \sin^2 \psi,
	& -r \sin \psi \cos \psi +r \sin \psi \cos \psi,
	& 0
\\
	-r \sin \psi \cos \psi +r \sin \psi \cos \psi,
	& r^2 \sin^2 \psi + r^2 \cos^2 \psi,
	& 0
\\
	0, & 0, &1
\end{array}
	\right|
\\
	= \left|
\begin{array}{ccc}
	1, & 0, & 0
\\
	0, & r^2, & 0
\\
	0, & 0, & 1
\end{array}
	\right|
\end{eqnarray*}

Il quadrato della lunghezza dell'elemento infinitesimale lineare $ ds^2 $ di una curva $ f(t) $ nel sistema $ x, y, z $ è quindi dato da
\[
	ds^2 = dx^2 + dy^2 + dz^2
\]
mentre nel sistema $ r,  \psi, z $
\[
	ds^2 = dr^2 + r^2 d\psi^2 + dz^2
\]

\end{document}