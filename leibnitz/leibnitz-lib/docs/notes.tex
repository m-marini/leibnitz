\documentclass[a4paper,twoside]{article}
\usepackage[T1]{fontenc}
\usepackage[utf8]{inputenc}
\usepackage{lmodern}
\usepackage[italian]{babel}
\usepackage{graphicx}
\begin{document}
\newtheorem{theorem}{Teorema}
\title{Leibnitz\thanks{$ $Id: notes.tex,v 1.3 2012/07/25 13:55:45 marco Exp $ $ }}
\author{Marco Marini}
\maketitle
\tableofcontents
\pagebreak

\section{Notes release 0.0.1.}

\subsection{Generals.}
The project computes the values of a vectorial function X defined by
a differential vectorial equation:
\begin{equation}
\vec X^{(n)} +\vec F(t, \vec X, \vec X^{(1)}, \dots, \vec X^{(n-1)} = 0
\end{equation}

given the initial Cauchy condition

\begin{equation}
\begin{array}{l}
\vec X(t_0) = \vec K
\\
\vec X^{(1)}(t_0) = \vec K_1
\\
\dots
\\
\vec X^{(n-1)}(t_0) = \vec K_{n-1}
\end{array}
\end{equation}

The $\vec X^{(n)}$ defines the derived $n$ function.

\subsection{Specifications.}

The syntax of function definition is defined in org.mmarini.leibnitz.parser.InterpreterContext javadoc
\\
\\
To use runs: java org.mmarini.leibintz.Compute file [-o file].
\\
The file parameter is a xml properties file containing the definitions.
\\
The output is stdout unless -o file is specified in line command.
\\

The mandatory properties in input file are
\begin{description}
	\item[order]: the order of function
	\item[dimension]: the space dimension
	\item[function]: the function $\vec F$
	\item[t0]: initial value of $t$
	\item[t1]: final value of $t$
	\item[dt]: increment of t value in single step
\end{description}

The optional properties are:
\begin{description}
	\item[x0 x1 ... xn-1]: Initial value of $\vec X$
	\item[x0(1) x1(1) ... xn-1(1)]: Initial value of $\vec X^{(1) }$
	\item[x0(o-1) x1(o-1) ... xn-1(o-1)]: Initial value of $\vec X^{(o-1) }$
\end{description}


\section{Addendum release 0.1.0.}

\begin{enumerate}
\item
Definire le funzioni di trasformazione tra le variabili indipendenti $\vec Q$ e le variabili reali $\vec R_\nu$.
Questo corrisponde con definire $m$ funzioni di trasformazione  $\vec R_\nu$ una per ogni particella.
\begin{equation}
	\vec R_\nu = \vec R_\nu(\vec Q)
\end{equation}

\item
Risolvere le equazioni di Lagrange in coordinate indipendenti $\vec Q$

Le equazioni di Lagrange sono definite come
\begin{equation}
\frac{d}{dt} \frac{\partial T}{\partial \dot q_i} - \frac{\partial T}{\partial q_i} = f_i
\end{equation}
$T$ è l'energia definita come
\begin{eqnarray}
T = T_2+T_1+T_0
\\
T_2 = a_{ij} \dot q_i \dot q_j
\\
T_1 =  a_i \dot q_i
\\
T_0 = a_0
\\
a_{ij} = m_\nu \frac{\partial \vec R_\nu}{\partial q_i} \frac{\partial \vec R_\nu}{\partial q_j}
\\
a_i  = m_\nu \frac{\partial \vec R_\nu}{\partial q_i} \frac{\partial \vec R_\nu}{\partial t}
\\
a_0  = m_\nu  \frac{\partial \vec R_\nu}{\partial t} \frac{\partial \vec R_\nu}{\partial t}
\end{eqnarray}
e $f_i$ sono le forze indipendenti (spostamenti virtuali)
\begin{equation}
f_i = \vec F_\nu \frac{\partial \vec R_\nu}{\partial q_i}
\end{equation}
ma abbiano che
\begin{eqnarray}
\frac{\partial T_2}{\partial \dot q_i}=a_{ij}\dot q_j
\\
\frac{\partial T_1}{\partial \dot q_i}=a_i
\\
\frac{\partial T_0}{\partial \dot q_i}=0
\\
\frac{\partial T}{\partial \dot q_i}=a_{ij}\dot q_j+a_i
\end{eqnarray}
derivando rispetto a $t$ abbiamo
\begin{equation}
\frac{d (a_{ij}\dot q_j+a_i)}{dt}
= a_{ij} \ddot q_j+\frac{d a_{ij}}{dt}\dot q_j+\frac{da_i}{dt}
= a_{ij} \ddot q_j+\left(\frac{\partial a_{ij}}{\partial q_j}\dot q_j+\frac{\partial a_{ij}}{\partial t}\right)\dot q_j+\frac{\partial a_i}{\partial q_j}\dot q_j+\frac{\partial a_i}{\partial t}
 \end{equation}
per cui
\begin{eqnarray}
a_{ij} \ddot q_j = G_i(t,q_j, \dot q_j)
\\
G_i=f_i+\frac{\partial T}{\partial q_i}-\left(\frac{\partial a_{ij}}{\partial q_j}\dot q_j-\frac{\partial a_{ij}}{\partial t}\right)\dot q_j-\frac{\partial a_i}{\partial q_j}\dot q_j-\frac{\partial a_i}{\partial t}
\end{eqnarray}
Quindi dobbiamo gestire vettori funzionali nello spazio $\Re^n$ e matrici funzionali nello spazio $\Re^{n \times n}$

Rappresentando in forma matriciale abbiamo:
\begin{equation}
A \vec {\ddot Q} = \vec G, \vec {\ddot Q} = A^{-1} \vec G
\end{equation}

\end{enumerate}

Le funzioni da implementare sono:

\begin{enumerate}
\item definizione di vettori e matrici funzionali di dimensione m (n. di gradi di libert?)
\item definizione di vettori funzionali di trasformazione di grado n (particelle)
\item estendere l'interpretazione e valutazione delle espressioni con matrici e con riferimenti a vettori e matrici funzionali.
\item variabili predefinite t,  Q, Q(1), ..., Q(n-1)
\end{enumerate}


\subsection{Specifications.}

The syntax of function definition is defined in org.mmarini.leibnitz.parser.SyntaxFactory javadoc
\\
\\
To use runs: java org.mmarini.leibintz.Compute file [-o file].
\\
The file parameter is a xml file containing the definitions.
\\
The output is stdout unless -o file is specified in line command.
\\

The file \textbf{definition} of xml is leibnitz-0.1.0.xsd


\section{Addendum release 0.4.0.}

Le condizioni iniziali sono impostate utilizzando il parse esteso di espressioni comprendenti anche la definizioni delle funzioni.
The file definition of xml is leibnitz-0.2.0.xsd


\subsection{Paraboloide.}

Calcoliamo il moto di una particella libera su una superficie paraboloide

\begin{eqnarray}
\vec R=(r \cos \varphi, r \sin \varphi, r^2)
\\
\frac{\partial \vec R}{\partial r }=(\cos \varphi,\sin \varphi, 2r)
\\
\frac{\partial \vec R}{\partial\varphi }=(-r \sin \varphi, r\cos \varphi, 0)
\end{eqnarray}

La forza agente sulla particella invece ?

\begin{equation}
\vec F = (0,0, -mg)
\end{equation}

da cui le forze generalizzate

\begin{eqnarray}
Q_r=\vec F \frac{\partial\vec R}{\partial r}=-2mgr
\\
Q_\varphi=\vec F \frac{\partial\vec R}{\partial \varphi}=0
\end{eqnarray}

Calcoliamo ora l'energia cinetica del sistema::

\begin{equation}
T=\frac{1}{2} m \frac{\partial \vec R}{\partial q_i}\frac{\partial \vec R}{\partial q_j} \dot q_i \dot q_j
=\frac{1}{2} m [(1+4r^2)\dot r^2+r^2\dot\varphi ^2]
\end{equation}

L'equazioni di Lagrange sono:

\begin{eqnarray}
\frac{d}{dt}\frac{\partial T}{\partial\dot r}-\frac{\partial T}{\partial r} = Q_r
\\
\frac{d}{dt}\frac{\partial T}{\partial\dot\varphi}-\frac{\partial T}{\partial\varphi} = Q_\varphi
\end{eqnarray}

\begin{eqnarray}
\frac{d}{dt}\frac{\partial T}{\partial\dot r}=\frac{d}{dt}[m(1+4r^2)\dot r]
=m[(1+4r^2)\ddot r+8r\dot r^2]
\\
\frac{\partial T}{\partial r}=4mr\dot r^2+mr\dot\varphi^2=mr(4\dot r^2+\dot\varphi^2)
\\
\frac{d}{dt}\frac{\partial T}{\partial\dot \varphi}=\frac{d}{dt}(mr^2\dot\varphi)
=mr^2\ddot\varphi+2mr\dot r\dot\varphi
\\
\frac{\partial T}{\partial \varphi}=0
\end{eqnarray}

quindi

\begin{equation}
\begin{array}{l}
\left\{
\begin{array}{l}
m(1+4r^2)\ddot r+8mr\dot r^2-4mr\dot r^2-mr\dot\varphi^2=-2mgr
\\
\\
mr^2\ddot\varphi +2mr\dot r\dot\varphi = 0
\end{array}
\right.
\\
\\
\left\{
\begin{array}{l}
\ddot r=\frac{r\dot\varphi^2-2gr-2r\dot r^2}{1+4r^2}
=r\frac{\dot\varphi^2-2g-2\dot r^2}{1+4r^2}
\\
\\
\ddot\varphi  = -2\frac{\dot r\dot\varphi}{r}
\end{array}
\right.
\end{array}
\end{equation}

Calcoliamo la velocit? angolare per mantenere la particella in una traiettoria circolare parallela al piano
$xOy$. E' necessario che la forza centripeta $Q_r$ sia esattamente opposta alla forza centrifuga
$mr\dot \varphi^2$ quindi:

\begin{equation}
2mgr=mr\dot\varphi^2 \Rightarrow \dot\varphi=\sqrt {2g}
\end{equation}


\subsection{Orbita su piano.}

Calcoliamo il moto di una particella libera su un piano con campo gravitazionale $\frac{K}{r^2}$

\begin{eqnarray}
\vec R=(r \cos \varphi, r \sin \varphi)
\\
\frac{\partial \vec R}{\partial r }=(\cos \varphi,\sin \varphi)
\\
\frac{\partial \vec R}{\partial\varphi }=(-r \sin \varphi, r\cos \varphi)
\end{eqnarray}

La forza agente sulla particella invece ?

\begin{equation}
\vec F = (-m\frac{K}{r^2}\cos\varphi, -m\frac{K}{r^2}\sin\varphi)
\end{equation}

da cui le forze generalizzate

\begin{eqnarray}
Q_r=\vec F \frac{\partial\vec R}{\partial r}=-m\frac{K}{r^2}
\\
Q_\varphi=\vec F \frac{\partial\vec R}{\partial \varphi}=0
\end{eqnarray}

Calcoliamo ora l'energia cinetica del sistema::

\begin{equation}
T=\frac{1}{2} m \frac{\partial \vec R}{\partial q_i}\frac{\partial \vec R}{\partial q_j} \dot q_i \dot q_j
=\frac{1}{2} m (\dot r^2+r^2\dot\varphi ^2)
\end{equation}

L'equazioni di Lagrange sono:

\begin{eqnarray}
\frac{d}{dt}\frac{\partial T}{\partial\dot r}-\frac{\partial T}{\partial r} = Q_r
\\
\frac{d}{dt}\frac{\partial T}{\partial\dot\varphi}-\frac{\partial T}{\partial\varphi} = Q_\varphi
\end{eqnarray}

\begin{eqnarray}
\frac{d}{dt}\frac{\partial T}{\partial\dot r}=\frac{d}{dt}[m\dot r]=m\ddot r
\\
\frac{\partial T}{\partial r}=mr\dot\varphi^2
\\
\frac{d}{dt}\frac{\partial T}{\partial\dot \varphi}=\frac{d}{dt}(mr^2\dot\varphi)
=mr^2\ddot\varphi+2mr\dot r\dot\varphi
\\
\frac{\partial T}{\partial \varphi}=0
\end{eqnarray}

quindi

\begin{equation}
\begin{array}{l}
\left\{
\begin{array}{l}
m\ddot r-mr\dot\varphi^2=-m\frac{K}{r^2}
\\
\\
mr^2\ddot\varphi +2mr\dot r\dot\varphi = 0
\end{array}
\right.
\\
\\
\left\{
\begin{array}{l}
\ddot r=r\dot \varphi^2-\frac{K}{r^2}
\\
\\
\ddot\varphi  = -2\frac{\dot r\dot\varphi}{r}
\end{array}
\right.
\end{array}
\end{equation}

Calcoliamo la velocità angolare per mantenere la particella in una traiettoria circolare. E' necessario che la forza centripeta $Q_r$ sia esattamente opposta alla forza centrifuga
$mr\dot \varphi^2$ quindi:

\begin{equation}
m \frac{K}{r^2} = m r \dot \varphi^2 \Rightarrow \dot \varphi = \frac{1}{r} \sqrt {\frac{K}{r}}
\end{equation}


\section{Addendum release 0.5.0.}

Se alcune delle variabili indipendenti rappresentano degli angoli allora il calcolo degli angoli di rotazione
deve essere eseguito con i quaternioni.
E' necessario quindi identificare quali variabili indipendenti rappresentano degli angoli.


\pagebreak
\appendix
\part{Appendice.}

\end{document}
