\documentclass[a4paper,11pt]{article}
\usepackage[T1]{fontenc}
\usepackage[utf8]{inputenc}
\usepackage{lmodern}
\usepackage[italian]{babel}
\usepackage{graphicx}
\newtheorem{theorem}{Teorema}

\title{Studio matematico per software Leibnitz}
\author{Marco Marini}

\begin{document}

\maketitle
\tableofcontents

\begin{abstract}
\end{abstract}

\part{Calcolo tensoriale}

\section{Tensore d'inerzia}

Prendiamo un sistema di coordinate cartesiane euclideo $x, y, z$.

Il tensore d'inerzia per un punto $ \vec r_0 = (x_0, y_0, z_0) $ è dato da
\begin{equation}
\begin{array}{r}
	I_{11} = \int_V \rho(\vec r) [(y - y_0)^2 + (z - z_0)^2 ] dV
\\
	I_{22} = \int_V \rho(\vec r) [(x - x_0)^2 + (z - z_0)^2 ] dV
\\
	I_{33} = \int_V \rho(\vec r) [(x - x_0)^2 + (y - y_0)^2 ] dV
\\
	I_{12} = I_{21} = -\int_V \rho(\vec r) (x - x_0) (y - y_0) dV
\\
	I_{13} = I_{31} = -\int_V \rho(\vec r) (x - x_0) (z - z_0) dV	
\\
	I_{23} = I_{31} = -\int_V \rho(\vec r) (y - y_0) (z - z_0) dV	
\end{array}
\end{equation}

Con il tensore di'inerzia si può calcolare sia il momento d'inerzia
\begin{equation}
	\vec L = L^i = I_{ij} \omega^j
\end{equation}

Sia l'energia cinetica di rotazione
\begin{equation}
	E = I_{ij} \omega^i \omega^j
\end{equation}


\subsection{Tensore d'inerzia di un parallelepipedo}

Come esempio calcoliamo il tensore d'inerzia di un parallelepipedo di di dimensione $ w \times h \times d $ e di densità uniforme $ \rho $ nel punto $ \vec {r_0} = (x_0, y_0, z_0) $ .

\begin{eqnarray*}
	I_{11} = \int_V \rho [ (y - y_0)^2 + (z - z_0)^2] dV
\\
	= \rho \int_{z=0}^h \int_{y=0}^d \int_{x=0}^w [ (y - y_0)^2 + (z - z_0)^2] dx \, dy \, dz
\\
	= \rho w \int \int [ (y - y_0)^2 + (z - z_0)^2] dy \, dz
\\
	= \rho w \int \left[ \frac{y^3}{3} - y^2 y_0 + y y_0 ^2 + y (z - z_0)^2 \right]_{y=0}^d dz
\\
	= \rho w d \int \left[ \frac{d^2}{3} - d y_0 + y_0 ^2 + (z - z_0)^2 \right] dz
\\
	= \rho w d \left[ \left(\frac{d^2}{3} - d y_0 + y_0 ^2\right) z + \frac{z^3}{3} - z^2 z_0 + z z_0^2 \right]_{z=0}^h
\\
	= \rho w d h \left( \frac{d^2}{3} - d y_0 + y_0 ^2 + \frac{h^2}{3} - h z_0 + z_0^2 \right)
\end{eqnarray*}
poichè $\rho w d h = M $ è la massa del parallelepipedo abbiamo
\begin{equation}
	I_{11} = M \left( \frac{d^2+h^2}{3} - d y_0 - h z_0 + y_0 ^2 + z_0^2 \right)
\end{equation}
in maniera simile si calcola
\begin{eqnarray}
	I_{22} = M \left( \frac{w^2+h^2}{3} - w x_0 - h z_0 + x_0 ^2 + z_0^2 \right)
\\
	I_{33} = M \left( \frac{w^2+d^2}{3} - w x_0 - d y_0 + x_0 ^2 + y_0^2 \right)
\end{eqnarray}

\begin{eqnarray*}
	I_{12} = -\int_V \rho (x - x_0) (y - y_0) dV
\\
	= -\rho \int_{z=0}^h \int_{y=0}^d \int_{x=0}^w (x - x_0) (y - y_0) dx \, dy \, dz
\\
	= -\rho \int \int (y - y_0)  \left. \left( \frac{x^2}{2} - x x_0 \right) \right|_{x=0}^w dy \, dz
\\
	= -\rho w \left( \frac{w}{2} - x_0 \right) \int \int (y - y_0) dy \,  dz
\\
	= -\rho w \left( \frac{w}{2} - x_0 \right) \int \left. \left( \frac{y^2}{2} - y y_0 \right) \right|_{y=0}^d  dz
\\
	= -\rho w d \left( \frac{w}{2} - x_0 \right) \left( \frac{d}{2} - y_0 \right) \int  dz
\\
	= -\rho w d h \left( \frac{w}{2} - x_0 \right) \left( \frac{d}{2} - y_0 \right)
\end{eqnarray*}
\begin{equation}
	I_{12} = -M \left( \frac{w}{2} - x_0 \right) \left( \frac{d}{2} - y_0 \right)
\end{equation}
similmente abbiamo
\begin{eqnarray*}
	I_{13} = -M \left( \frac{w}{2} - x_0 \right) \left( \frac{h}{2} - z_0 \right)
\\
	I_{23} = -M \left( \frac{d}{2} - y_0 \right) \left( \frac{h}{2} - z_0 \right)
\end{eqnarray*}

In sintensi
\[
	I_{ij} = M
	\left|
\begin{array}{ccc}
	\frac{d^2+h^2}{3} - d y_0 - h z_0 + y_0 ^2 + z_0^2 ,
	& - \left( \frac{w}{2} - x_0 \right) \left( \frac{d}{2} - y_0 \right),
	& - \left( \frac{w}{2} - x_0 \right) \left( \frac{h}{2} - z_0 \right)
\\
	-\left( \frac{w}{2} - x_0 \right) \left( \frac{d}{2} - y_0 \right), 
	& \frac{w^2+h^2}{3} - w x_0 - h z_0 + x_0 ^2 + z_0^2,
	& - \left( \frac{d}{2} - y_0 \right) \left( \frac{h}{2} - z_0 \right)
\\
	-\left( \frac{w}{2} - x_0 \right) \left( \frac{h}{2} - z_0 \right),
	& -\left( \frac{d}{2} - y_0 \right) \left( \frac{h}{2} - z_0 \right),
	&  \frac{w^2+d^2}{3} -w x_0 - d y_0 + x_0 ^2 + y_0^2
\end{array}
	\right|
\]

Si può facilmente vedere che nel caso si prenda come punto di rotazione il centro di massa
$ \vec r_0 = \left( \frac{w}{2}, \frac{d}{2}, \frac{h}{2} \right) $ 
la matrice $ I_{ij}$ si diagonalizza in
\[
	I_{ij} = \frac{M}{12}
	\left|
\begin{array}{ccc}
	d^2 + h^2, & 0, & 0
\\
	0, & w^2 + h^2, & 0
\\
	0, & 0, & w^2 + d^2
\end{array}
	\right|
	=
	\left|
\begin{array}{ccc}
	I_1, &0, &0
\\
	0, & I_2, & 0
\\
	0, & 0, & I_3
\end{array}
	\right|
\]
con
\begin{eqnarray*}
	I_1 = \frac{M}{12}(d^2 + h^2)
\\
	I_2 = \frac{M}{12}(w^2 + h^2)
\\
	I_3 = \frac{M}{12}(w^2 + d^2) 
\end{eqnarray*}
e gli assi $ x, y, z $ risultano essere gli assi d'inerzia principali.


\subsection{Esempio di cambio di sistema di riferimento}

Prendiamo ora in cosiderazione un esempio di cambio di sistema di riferimento per osservare come
varia il tensore d'inerzia.

Prendiamo in considerazione il centro di massa del parallelepipedo $\vec r_0 = \left( \frac{w}{2},  \frac{d}{2},  \frac{h}{2} \right) $
\[
	I_{ij} =
	\left|
\begin{array}{ccc}
	I_1, &0, &0
\\
	0, & I_2, & 0
\\
	0, & 0, & I_3
\end{array}
	\right|
\]

Prendiamo ora il sistema di riferimento $ z^1, z^2, z^3 $
\begin{eqnarray*}
	z^1 = x^1 \cos \upsilon t - x^2 \sin \upsilon t
\\
	z^2 = x^1 \sin \upsilon t + x^2 \cos \upsilon t
\\	
	z^3 = x^3
\end{eqnarray*} che corrisponde ad un sistema in rotazione attorno all'asse $z$ alla velocità angolare $ \upsilon $.

Lo jacobiano è
\[
	J^i_j = \frac{\partial z^i}{\partial x^j} = 
\left|
\begin{array}{lll}
	\cos \upsilon t, & -\sin \upsilon t, & 0
\\
	\sin \upsilon t, & \cos \upsilon t, & 0
\\
	0, & 0, & 1
\end{array}
\right|
	=
\left|
\begin{array}{lll}
	c, & -s, & 0
\\
	s, & c, & 0
\\
	0, & 0, & 1
\end{array}
\right|
\]
mentre l'inveso è 
\[
	(J^{-1})^i_j = \frac{\partial x^i}{\partial z^j} = 
\left|
\begin{array}{lll}
	c, & s, & 0
\\
	-s, & c, & 0
\\
	0, & 0, & 1
\end{array}
\right|
\]

Il tensore d'inerzia diventa
\begin{eqnarray*}
	I'_{ij} = I_{kl} \frac{\partial x^k}{\partial z^i} \frac{\partial x^l}{\partial z^j}
	= I_{kl} (J^{-1})^k_i (J^{-1})^l_j = (J^{-1})^T I J^{-1}
\\
	= \left|
\begin{array}{lll}
	c, & -s, & 0
\\
	s, & c, & 0
\\
	0, & 0, & 1
\end{array}
	\right|
	\left|
\begin{array}{lll}
	I_1, & 0, & 0
\\
	0, & I_2, & 0
\\
	0, & 0, & I_3
\end{array}
	\right|
	\left|
\begin{array}{lll}
	c, & s, & 0
\\
	-s, & c, & 0
\\
	0, & 0, & 1
\end{array}
\right|
\\
=
	\left|
\begin{array}{lll}
	c^2 I_1 + s^2 I_2, & sc (I_1 - I_2), & 0
\\
	sc (I_1 - I_2), &s^2 I_1 + c^2 I_2 , & 0
\\
	0, & 0, & I_3
\end{array}\
\right|
\end{eqnarray*}

Se il parallelepipedo ruota ad una velocià angolare $ \vec \omega = \omega^i \vec e_i$ il momento d'inerzia è dato da
\[
	\vec L = L_i \vec {e^i} = I_{ij} \omega^j \vec {e^i} = I_1 \omega^1 \vec {e^1} +  I_2 \omega^2 \vec {e^2} + I_3 \omega^3 \vec {e^3}
\]
mentre nel sistema di coordinate $ z^1,  z^2, z^3 $
la velocità angolare è data da
\[
	{\omega'}^i = \omega^j \frac{\partial z^i}{\partial x^j} = J \vec w = 
	\left|
\begin{array}{lll}
	c, & -s, & 0
\\
	s, & c, & 0
\\
	0, & 0, & 1	
\end{array}
	\right|
	\left|
\begin{array}{l}
	\omega^1
\\
	\omega^2
\\
	\omega^3
\end{array}
	\right|
=
	\left|
\begin{array}{l}
	c \omega^1 - s \omega^2
\\
	s \omega^1 + c \omega^2
\\
	\omega^3
\end{array}
	\right|
\]
mentre il tensore d'inerzia è
\begin{eqnarray*}
	\vec L' = L'_i \vec {e'^i} = I'_{ij} {\omega'}^j \vec {e'^i}
\\
	= 
	\left|
\begin{array}{lll}
	c^2 I_1 + s^2 I_2, & sc (I_1 - I_2), & 0
\\
	sc (I_1 - I_2), &s^2 I_1 + c^2 I_2 , & 0
\\
	0, & 0, & I_3
\end{array}
	\right|
	\left|
\begin{array}{l}
	c \omega^1 - s \omega^2
\\
	s \omega^1 + c \omega^2
\\
	\omega^3
\end{array}
	\right|
= 
	\left|
\begin{array}{l}
	c I_1 \omega^1 - s I_2 \omega^2
\\
	s I_1 \omega^1 + c I_2 \omega^2
\\
	I_3 \omega^3
\end{array}
	\right|
\end{eqnarray*}
verifichiamo che applicando la trasformazione di coordinate il momento d'inerzia risulta uguale a quanto calcolato:
\begin{eqnarray*}
	L'_i = L_j \frac{\partial x^j}{\partial z^i} = ( J^{-1} )^T L
	= 
	\left|
\begin{array}{lll}
	c, & -s, & 0
\\
	s, &c, & 0
\\
	0, & 0, & 1
\end{array}
	\right|
	\left|
\begin{array}{l}
	I_1 \omega^1
\\
	I_2 \omega^2
\\
	I_1 \omega^3
\end{array}
	\right|
= 
	\left|
\begin{array}{l}
	c I_1 \omega^1 - s I_2 \omega^2
\\
	s I_1 \omega^1 + c I_2 \omega^2
\\
	I_3 \omega^3
\end{array}
	\right|
\end{eqnarray*}


\subsection{Dinamica}

L'equazione dinamica del moto rotazionale è
\[
	\frac{d L_i}{dt} = M_i
\]
dove $ M_i $ è la somma dei momenti delle forze agenti sul corpo.
\[
	M_i = \frac{d L_i}{dt} = \frac{d I_{ij} \omega^j}{dt};
\]

Quindi posto
\[
\dot \omega^j = \frac{d  \omega^j}{dt}
\]
abbiamo
\begin{equation}
	M_i = \frac{d I_{ij}}{dt} \omega^j + I_{ij} \dot \omega^j 
\end{equation}
\\

Nel sistema $ x^1, x^2, x^3 $ abbiamo che $ I_{ij} $ non dipende esplicitamente da $ t $ quindi $ \frac{d I_{ij}}{dt} = 0 $ da cui
\[
	M_i = I_{ij} \dot \omega^j
\]
da cui
\[
	I^{-1}_{ki} M_i = I^{-1}_{ki} I_{ij} \dot \omega^j = \dot \omega^k
\]
In caso di corpo in equilibrio le forze agenti sul corpo sono nulle quindi $ M_i = 0 $ quindi 
\[
	\dot \omega^i = 0 \Rightarrow \omega^i = \omega_0^i
\]
e il corpo permane i rotazione costante sull'asse in direzione $ \vec \omega_0 $
\\

Nel sistema $ z^1, z^2, z^3 $ abbiamo
\[
	M'_i = \frac{dL'_i}{dt} = \frac{d I'_{ij}}{dt} {\omega'}^j + I'_{ij} \dot \omega'^j
\]
che in caso di equilibrio è $ M'_i = ({J^{-1}})_i^j M_j = 0 $
\[
	\frac{d I'_{ij}}{dt} {\omega'}^j + I'_{ij} \dot \omega'^j = 0
\]
quindi 
\[
	I'_{ij} \dot \omega'^j = -\frac{d I'_{ij}}{dt} {\omega'}^j
\]
da cui possiamo ricavare
\begin{equation}
	\dot \omega'^k = -I_{ki}^{'-1} \frac{d I'_{ij}}{dt} {\omega'}^j
\end{equation}

Calcoliamo
\begin{eqnarray*}
	\frac{d I'_{ij}}{dt} = \frac{d (J^{-1})^T I J^{-1}}{dt}
\\
	= \frac{d (J^{-1})^T}{dt} I J^{-1} + (J^{-1})^T \frac{d I}{dt} J^{-1} + (J^{-1})^T I \frac{d J^{-1}}{dt}
\\
	=  \frac{d (J^{-1})^T}{dt} I J^{-1} + (J^{-1})^T I \frac{d J^{-1}}{dt}
\end{eqnarray*}

\subsection{Dinamica 1}

Prendiamo il sistema di riferimento $ x^1, x^2, x^3 $, abbiamo visto che in tale sistema l'equazione del moto 
risulta essere

\[
	\dot \omega^k = I_ {ki}^{-1} M_i
\]

Ipotiziamo che $ M_i $ sia dovuto alle forze $ \vec F_1 = (F, 0, 0) $ e $ \vec F_2 = (-F, 0, 0) $ applicate ai rispettivi punti
$P_1 = (0, 0, h) $ e $ P_2 = (0, 0, 0) $

Il momento risultate delle forze sarà quindi 

\begin{equation}
M_i = \sum F_\nu^i \times (P_\nu^i - R_0^i) =
\
\begin{array}{lll}

\end{array}
\end{equation}

\section{Appendice di calcolo}
Matrice Jacobiana
\begin{equation}
	J =
\left|
\begin{array}{lll}
	c,	&	-s,	&	0
	\\
	s,	&	c,	&	0
	\\
	0,	&	0,	&	1
\end{array}
\right|
\end{equation}

Matrice Jacobiana inversa
\begin{equation}
	J^{-1} =
\left|
\begin{array}{lll}
	c,	&	s,	&	0
	\\
	-s,	&	c,	&	0
	\\
	0,	&	0,	&	1
\end{array}
\right|
\end{equation}

Derivata della jacobiana inversa:
\[
	\frac{d (J^{-1})^i_j}{dt} = \upsilon
	\left|
\begin{array}{lll}
	-s, & c, & 0
\\
	-c, &-s, & 0
\\
	0, & 0, & 0
\end{array}
	\right|
\]

Tensore d'inerzia nel sistema $ x^1, x^2, x^3 $ 
\begin{equation}
	I =
\left|
\begin{array}{lll}
	I_1,	&	0,	&	0
	\\
	0,	&	I_2,	&	0
	\\
	0,	&	0,	&	I_3
\end{array}
\right|
\end{equation}

Inversa del tensore d'inerzia nel sistema $ x^1, x^2, x^3 $ 
\begin{equation}
	I^{-1} =
\left|
\begin{array}{lll}
	\frac{1}{I_1},	&	0,			&	0
	\\
	0,			&	\frac{1}{I_2},	&	0
	\\
	0,			&	0,			&	\frac{1}{I_3}	
\end{array}
\right|
\end{equation}


Tensore d'inerzia nel sistema $ z^1, z^2, z^3 $ 
\begin{eqnarray*}
	I' = (J^{-1})^T I J^{-1} =
\left|
\begin{array}{lll}
	c,	&	-s,	&	0
	\\
	s,	&	c,	&	0
	\\
	0,	&	0,	&	1
\end{array}
\right|
\left|
\begin{array}{lll}
	I_1,	&	0,	&	0
	\\
	0,	&	I_2,	&	0
	\\
	0,	&	0,	&	I_3
\end{array}
\right|
\left|
\begin{array}{lll}
	c,	&	s,	&	0
	\\
	-s,	&	c,	&	0
	\\
	0,	&	0,	&	1
\end{array}
\right|
\\
=\left|
\begin{array}{lll}
	c I_1,	&	-s I_2,	&	0
	\\
	s I_1,	&	c I_2,	&	0
	\\
	0,		&	0,		&	I_3
\end{array}
\right|
\left|
\begin{array}{lll}
	c,	&	s,	&	0
	\\
	-s,	&	c,	&	0
	\\
	0,	&	0,	&	1
\end{array}
\right|
=\left|
\begin{array}{lll}
	c^2 I_1 + s^2 I_2,	&	s c (I_1 - I_2),		&	0
	\\
	s c (I_1 - I_2),		&	s^2 I_1 + c^2 I_2,	&	0
	\\
	0,				&	0,				&	I_3
\end{array}
\right|
\end{eqnarray*}

Derivata temporale del tensore d'inerzia in $ z^1, z^2, z^3 $
\begin{eqnarray}
	\frac{d I'}{dt}
	= \left|
	\begin{array}{lll}
	2 c (-s) \upsilon I_1+ 2 s c \upsilon I_2  ,	&	\upsilon (c^2-s^2) (I_1 - I_2),		&	0
	\\
	\upsilon (c^2-s^2) (I_1 - I_2),			&	2 c s \upsilon I_1- 2 s c \upsilon I_2,	&	0
	\\
	0,								&	0,							&	0
	\end{array}
	\right|
	\\
	= \upsilon (I_1 - I_2) \left|
	\begin{array}{lll}
	-2 s c  ,		&	c^2 - s ^2,	&	0
	\\
	c^2 - s ^2,	&	2 s c,			&	0
	\\
	0,			&	0,			&	0
	\end{array}
	\right|
\end{eqnarray}

Inversa del tensore d'inerzia in $ z^1, z^2, z^3 $
\[
	{I'}^{-1} = 
\left| 
\begin{array}{lll}
	c^2 I_1 + s^2 I_2,	&	sc (I_1 - I_2),		&	0
\\
	sc (I_1 - I_2),		&	s^2 I_1 + c^2 I_2,	&	0
\\
	0,				&	0,				&	I_3
\end{array}
\right|^{-1}
=
\left| 
\begin{array}{lll}
	a,	&	b,	&	0
\\
	b,	&	d,	&	0
\\
	0,	&	0,	&	e
\end{array}
\right|^{-1}
\]


\begin{eqnarray*}
\left| 
\begin{array}{lll}
	a,	&	b,	&	0
\\
	b,	&	d,	&	0
\\
	0,	&	0,	&	e
\end{array}
\right|
\left.
\begin{array}{lll}
	1,	&	0,	&	0
\\
	0,	&	1,	&	0
\\
	0,	&	0,	&	1
\end{array}
\right|
\Rightarrow
\left| 
\begin{array}{lll}
	ab,	&	b^2,	&	0
\\
	ab,	&	ad,	&	0
\\
		&	\cdots	&
\end{array}
\right|
\left.
\begin{array}{lll}
	b,	&	0,	&	0
\\
	0,	&	a,	&	0
\\
		&	\cdots	&
\end{array}
\right|
\\
\left| 
\begin{array}{lll}
		&	\cdots	&
\\
	0,	&	b^2-ad,	&	0
\\
		&	\cdots	&
\end{array}
\right|
\left.
\begin{array}{lll}
		&	\cdots	&
\\
	b,	&	-a,	&	0
\\
		&	\cdots	&
\end{array}
\right|
\Rightarrow
\left| 
\begin{array}{lll}
		&	\cdots	&
\\
	0,	&	1,	&	0
\\
		&	\cdots	&
\end{array}
\right|
\left.
\begin{array}{lll}
		&	\cdots	&
\\
	\frac{b}{b^2 - ad},	&	-\frac{a}{b^2 - ad},	&	0
\\
		&	\cdots	&
\end{array}
\right|
\\
\left| 
\begin{array}{lll}
	ad,	&	bd,	&	0
\\
	b^2,	&	bd,	&	0
\\
		&	\cdots	&
\end{array}
\right|
\left.
\begin{array}{lll}
	d,	&	0,	&	0
\\
	0,	&	b,	&	0
\\
		&	\cdots	&
\end{array}
\right|
\Rightarrow
\left| 
\begin{array}{lll}
	b^2 - ad,	&	0,	&	0
\\
		&	\cdots	&
\\
		&	\cdots	&
\end{array}
\right|
\left.
\begin{array}{lll}
	-d,	&	b,	&	0
\\
		&	\cdots	&
\\
		&	\cdots	&
\end{array}
\right|
\\
\left| 
\begin{array}{lll}
	1,	&	0,	&	0
\\
		&	\cdots	&
\\
		&	\cdots	&
\end{array}
\right|
\left.
\begin{array}{lll}
	-\frac{d}{b^2 - ad},	&	\frac{b}{b^2 - ad},	&	0
\\
		&	\cdots	&
\\
		&	\cdots	&
\end{array}
\right|
\Rightarrow
\left| 
\begin{array}{lll}
	1,	&	0,	&	0
\\
	0,	&	1,	&	0
\\
	0,	&	0,	&	1
\end{array}
\right|
\left.
\begin{array}{lll}
	-\frac{d}{b^2 - ad},	&	\frac{b}{b^2 - ad},	&	0
\\
	\frac{b}{b^2 - ad},	&	-\frac{a}{b^2 - ad},	&	0
\\
	0	,			&	0,				&	\frac{1}{e}
\end{array}
\right|
\\
b^2 - ad = s^2 c^2 (I_1 - I_2)^2 - (c^2 I_1 + s^2 I_2) (s^2 I_1 + c^2 I_2) = 
\\
s^2 c^2 I_1^2 + s^2 c^2 I_2^2 - 2 s^2 c^2 I_1 I_2 - s^2 c^2 I_1^2 - s^2 c^2 I_2^2 - I_1 I_2 (s^4 + c^4) =
\\
- I_1 I_2 (s^4 + c^4 + 2 s^2 c^2) = - I_1 I_2 (s^2 + c^2)^2 = - I_1 I_2 
\\
\end{eqnarray*}
\begin{equation}
{I'}^{-1} = 
\left|
\begin{array}{lll}
	\frac{s^2 I_1 + c^2 I_2}{I_1 I_2},	&	-sc \frac{I_1 - I_2}{I_1 I_2},		&	0
\\
	-sc \frac{I_1 - I_2}{I_1 I_2},		&	\frac{c^2 I_1 + s^2 I_2}{I_1 I_2},	&	0
\\
	0	,						&	0,							&	\frac{1}{I_3}
\end{array}
\right|
\end{equation}

Accelerazione angolare nel sistema $ z^1, z^2, z^3 $

\begin{eqnarray*}
	\dot \omega'^k = - {I'_{ki}}^{-1} \frac{d I'_{ij}}{dt} {\omega'}^j
\\
	= - \upsilon (I_1 - I_2)
	\left|
	\begin{array}{lll}
	\frac{s^2 I_1 + c^2 I_2}{I_1 I_2},	&	-sc \frac{I_1 - I_2}{I_1 I_2},		&	0
	\\
	-sc \frac{I_1 - I_2}{I_1 I_2},		&	\frac{c^2 I_1 + s^2 I_2}{I_1 I_2},	&	0
	\\
	0,							&	0,							&	\frac{1}{I_3}
	\end{array}
	\right|
	\left|
	\begin{array}{lll}
	-2 s c,	&	c^2 - s^2,	&	0
	\\
	c^2 - s^2,	&	2 s c,		&	0
	\\
	0,		&	 0,		&	0
	\end{array}
	\right|
	\left|
	\begin{array}{l}
	{\omega'}^1
	\\
	{\omega'}^2
	\\
	{\omega'}^3
	\end{array}
	\right|
\\
	= \upsilon \frac{I_1 - I_2}{I_1 I_2}
	\left|
	\begin{array}{lll}
	a,	&	b,	&	0
	\\
	f,	&	g,	&	0
	\\
	0,	&	0,	&	0
	\end{array}
	\right|
	\left|
	\begin{array}{l}
	{\omega'}^1
	\\
	{\omega'}^2
	\\
	{\omega'}^3
	\end{array}
	\right|
\\
\\
	a = s c [2 (s^2 I_1 + c^2 I_2) + (I_1 - I_2) (c^2 - s^2)]
\\
	= s c (2 s^2 I_1 + 2 c^2 I_2 + c^2 I_1 - s^2 I_1 - c^2 I_2 + s^2 I_2)
\\
	= s c (s^2 I_1 + c^2 I_2 + c^2 I_1 + s^2 I_2) = s c (I_1 + I_2)
\\
\\
	b = -(s^2 I_1 + c^2 I_2) (c^2 - s^2) + 2 s^2 c^2 (I_1 - I_2)
\\
	= -s^2 c^2 I_1 + s^4 I_1 - c^4 I_2 + s^2 c^2 I_2 + 2 s^2 c^2 I_1 - 2 s^2 c^2 I_2
\\
	= s^2 c^2 I_1 + s^4 I_1 - c^4 I_2 - s^2 c^2 I_2
	= s^2 (c^2 + s^2) I_1 - c^2 (c^2 + s^2) I_2
\\
	= s^2 I_1 - c^2 I_2
\\
\\
	f = -2 s^2 c^2 (I_1 - I_2) - (c^2 I_1 + s^2 I_2) (c^2 - s^2) 
\\
	= -2 s^2 c^2 I_1 + 2 s^2 c^2 I_2 - c^4 I_1 + s^2 c^2 I_1 - s^2 c^2 I_2 + s^4 I_2
\\
	= - s^2 c^2 I_1 + s^2 c^2 I_2 - c^4 I_1 + s^4 I_2
	= -c^2 (s^2 + c^2) I_1 + s^2 (c^2 + s^2) I_2
\\
	= -c^2 I_1 + s^2 I_2
\\
\\
	g = s c [(I_1 - I_2) (c^2 - s^2) - 2 (c^2 I_1 + s^2 I_2)]
\\
	= s c (c^2 I_1 - s^2 I_1 - c^2 I_2 + s^2 I_2 - 2 c^2 I_1 - 2 s^2 I_2) 
\\
	= s c (-c^2 I_1 - s^2 I_1 - c^2 I_2 - s^2 I_2)
	= - s c (I_1 + I_2)
\\
\\
	\left|
	\begin{array}{l}
	\dot {\omega '}^1
	\\
	\dot {\omega'}^2
	\\
	\dot {\omega'}^3
	\end{array}
	\right| =
	\upsilon \frac{I_1 - I_2}{I_1 I_2}
	\left|
	\begin{array}{lll}
	s c (I_1 + I_2),		&	s^2 I_1 - c^2 I_2,	&	0
	\\
	-c^2 I_1 + s^2 I_2,	&	-s c (I_1 + I_2),		&	0
	\\
	0,				&	0,				&	0
	\end{array}
	\right|	
	\left|
	\begin{array}{l}
	{\omega '}^1
	\\
	{\omega'}^2
	\\
	{\omega'}^3
	\end{array}
	\right|
\end{eqnarray*}

\end{document}